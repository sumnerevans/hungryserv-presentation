\documentclass{beeper}

\usepackage{fontawesome}
\usepackage{etoolbox}
\usepackage{textcomp}
\usepackage[nodisplayskipstretch]{setspace}
\usepackage{xspace}
\usepackage{verbatim}
\usepackage{multicol}
\usepackage{soul}
\usepackage{attrib}

\usepackage{amsmath,amssymb,amsthm}

\usepackage[linesnumbered,commentsnumbered,ruled,vlined]{algorithm2e}
\newcommand\mycommfont[1]{\footnotesize\ttfamily\textcolor{blue}{#1}}
\SetCommentSty{mycommfont}
\SetKwComment{tcc}{ \# }{}
\SetKwComment{tcp}{ \# }{}

\usepackage{siunitx}

\usepackage{tikz}
\usepackage{pgfplots}
\usetikzlibrary{decorations.pathreplacing,calc,arrows.meta,shapes,graphs}

\AtBeginEnvironment{minted}{\singlespacing\fontsize{10}{10}\selectfont}
\setmainfont{Open Sans Light}
\usefonttheme{serif}

\makeatletter
\patchcmd{\beamer@sectionintoc}{\vskip1.5em}{\vskip0.5em}{}{}
\makeatother

% Math stuffs
\newcommand{\Z}{\mathbb{Z}}
\newcommand{\R}{\mathbb{R}}
\newcommand{\N}{\mathbb{N}}
\newcommand{\lcm}{\text{lcm}}
\newcommand{\Inn}{\text{Inn}}
\newcommand{\Aut}{\text{Aut}}
\newcommand{\Ker}{\text{Ker}\ }
\newcommand{\la}{\langle}
\newcommand{\ra}{\rangle}

\newcommand{\yournewcommand}[2]{Something #1, and #2}

\newenvironment{question}[1]{\par\textbf{Question #1.}\par}{}

\newcommand{\pmidg}[1]{\parbox{\widthof{#1}}{#1}}
\newcommand{\splitslide}[4]{
    \noindent
    \begin{minipage}{#1 \textwidth - #2 }
        #3
    \end{minipage}%
    \hspace{ \dimexpr #2 * 2 \relax }%
    \begin{minipage}{\textwidth - #1 \textwidth - #2 }
        #4
    \end{minipage}
}

\newcommand{\frameoutput}[1]{\frame{\colorbox{white}{#1}}}

\newcommand{\tikzmark}[1]{%
\tikz[baseline=-0.55ex,overlay,remember picture] \node[inner sep=0pt,] (#1)
{\vphantom{T}};
}

\newcommand{\braced}[3]{%
    \begin{tikzpicture}[overlay,remember picture]
        \draw [thick,decorate,decoration={brace,raise=1ex,amplitude=4pt},blue] (#2.south west-|T1.south west) -- node[anchor=west,left,xshift=-1.8ex,text=olive]{#3} (#1.north west-|T1.south west);
    \end{tikzpicture}
}

\title{Hungryserv}
\subtitle{A Homeserver Optimized for Unfederated Use-Cases}
\author{Sumner Evans}
\date{27 August 2022}

\begin{document}

\begin{frame}{A bit about me}
    My name is Sumner, I'm a software engineer at Beeper.
    \begin{itemize}
        \item I graduated from Colorado School of Mines in 2019 with a master's
            in CS.
        \item I teach as an adjunct professor at my alma mater.
        \item I enjoy skiing, volleyball, and football (soccer).
        \item I'm a 4th degree black belt in ATA taekwondo.
    \end{itemize}
    \pause

    I became interested in Matrix when I was the chair of the ACM chapter at
    Mines. I was looking for an open source chat platform for the club to use,
    and Matrix fit the bill!
\end{frame}

\begin{frame}{What I work on at Beeper}
    I am on the newly created \textit{Scaling} team.

    Our current objective is to prepare Beeper for rocket-ship growth.
    \pause

    I was previously part of the \textit{Bridges} team. 

    Notable projects include:
    \begin{itemize}
        \item Writing the LinkedIn bridge
        \item Implementing massive stability improvements to the Signal bridge.
        \item Implementing incremental infinite backfill in our WhatsApp and
            Facebook bridges
    \end{itemize}
\end{frame}

\begin{frame}{Overview}
    \setbeamertemplate{section in toc}[sections numbered]
    \tableofcontents[hideallsubsections]

    \pause
    \textbf{If you have questions at any point, feel free to interrupt me.}
\end{frame}

\section{A bit about Beeper}

\begin{frame}{Beeper's mission}
    Our mission is to:

    \begin{quote}
        make it easy for everyone on Earth to chat with each other.
    \end{quote}
    \pause

    We specifically chose the word ``chat'' rather than ``communicate'' because
    we are focusing on \textit{people talking to one another}.
\end{frame}

\begin{frame}{How we are getting there}
    % focusing on DMs not group chats
    % bring all your chats to the same place - switch with no cost
\end{frame}

\section{A bit about Beeper's current architecture}

\begin{frame}{We run a lot of bridges}
    We encourage users to connect \textbf{all} of their chat networks.
    Each user gets their own bridge for each network they connect.

    % Advantage: don't affect other users
    % can have different features for different sets of users
\end{frame}

\begin{frame}{A diagram}
    % show an abbreviated diagram of the architecture
\end{frame}

\begin{frame}{Stats}
    % stats about the number of puppets per actual user

    None of that traffic is federated, but it has to go to Synapse which is
    designed with federation front-and-center!
\end{frame}

\begin{frame}[standout]
    \Large
    Solution: build a homeserver dedicated to unfederated traffic
\end{frame}

\section{Hungryserv}

\begin{frame}{Why is it hungry?}
    \Large
    Because it's unfed(erated)!
\end{frame}

\begin{frame}{Our new architecture}

\end{frame}

% When your DAG is a linked list, don't store it as a DAG

\end{document}
% Local Variables:
% TeX-command-extra-options: "-shell-escape"
% End:
